\documentclass{article}
\usepackage[utf8]{inputenc}
\begin{document}
\textbf{Introduction:} This group of people are aged 18-35 and are trying to adapt to your adulthood and take up the responsibilities of an adult. They comprises of a large part of the population of villages. They also have very high aspirations but even after being capable, they miss out of doing something big due to the lack of opportunities in the rural areas. \\\\
\textbf{Some experiences}\\
\textbf{CASE 1}\\
Name: Lakhan \\
Age: 25\\
Family members: 2 brothers(both older), 2 parents\\
Financial Conditions: lower-middle class, father was a farmer but can not work anymore so all 3 brothers work in the farm to earn . Due to the division of land, none of the brothers have substantial piece of land to make a living. Being the youngest in the family, lakhan was encouraged to complete his studies and helped by his brothers for the same. Despite being fairly educated, he is unable to find jobs in his village.\\
Daily routine:\\ 
6:00 - 6:30 Wakes up and have breakfast, which generally comprises of chapati made the night earlier\\
6:30 - 9:30 Tends to his farm, provide water and fertilizer to them.\\
9:30 - 3:30 goes to a tuition center where he takes tuition for children of the village.\\
3:30 - 4:30 has lunch\\
4:30 - 6:30 goes back to his farm and help his brothers\\
6:30 - 9:30 has dinner with his family and relax a bit\\
9:30 goes to bed\\
Spending: \\
1) food for the whole family, medication for his parents\\
2) raw material for his crops\\
3) personal costs, e.g. his parents are big believers of satsangs and they pay the donations regularly even if they can not afford to\\
Earning:\\
1) The money earned by selling the crops, if they survive the season\\
2) The money he gets from teaching\\
Aspirations:\\
He wants to be able to provide for his family sufficiently in the short term. In the long term, he wants to become a mechanical engineer and work in an automotive firm.\\
Most Important Resource\\
His degree, as it separates him from all the uneducated youth in the village. \\
What he lacks?\\
The opportunities are not available in the village. The best he can do after becoming an engineer is to teach tuition. This has forced him to give up his dream and become a farmer.\\\\\\
\textbf{CASE 2}\\
Name: Ram\\
Age: 21\\
Family members: Younger sister, mother\\
Financial Conditions: 
Lower middle class; His father passed away when he was only 10 years old.
His mother, who works as a housekeeper, was the sole earner in the house till last year. Lakhan is currently pursuing his bachelor's degree from a prestigious institute under the ST quota despite a lack of formal education. He also has a part-time job as a teacher to help students prepare for their entrance exam.\\
Daily routine:\\
7:00 - 8:00 Wakes up, has breakfast, and gets ready for his day\\
8:00 - 12:00 Attends his classes on the laptop provided by his college while sitting in his hostel room\\
12:00 - 12:30 Has his lunch in the hostel mess\\
12:30 - 2:30 Records lectures for his part-time job and answers the queries of his students through discussion forums online\\
2:30 - 6:30 Relaxes a bit for a while then goes out to play with his friends\\
6:30 - 8:30 Prepares for his job interviews starting next year by polishing his skill set\\
8:30 - 10:00 Has dinner and revises the material taught to him \\
10:00 Goes to bed\\
Spending:\\
1) He uses the stipend of his part-time job to pay off his tuition and hostel fees.\\
2) The remainder of his fees is sent as remittance to his mother.\\
Earning:\\
His sole source of earning is via his part-time job by teaching kids.\\
Aspirations:\\
Having lived a life of poverty, he wishes to bag a high paying tech job so he can financially support his family and pay off his younger sister's dowry.\\
Most Important Resource:\\
Just like Lakhan, he also has a degree which makes opens up more opportunities for him. Also, his eagerness to earn is fueled by his family's poor background.\\
What does he lack?\\
He doesn't have a financially stable family which severely hinders his ability to take risks in the future. Getting a well-paying job is a necessity for him, not just an aspiration. \\\\
\newpage
\textbf{CASE 3}\\
Name: Kriti\\
Age: 18\\
Family: Lives with Husband and in-laws. Parents, younger sister in
maiden home.\\
Background: She was bright at school. Parents, although caring and
supportive throughout childhood gave in to social pressure and forced her
to marry immediately after high school, at age 16. They also had to give
the in-laws their tractor as Dowry.\\
Financial condition: lower middle class, husband has a drinking and gambling problem, and spends most of savings on booze and betting. Kriti has
to work side jobs like sewing, growing and vending vegetables and flowers,
and helping her husband on the 1.5 acre family farm to keep the family
fed.\\
Social condition: Working women in rural areas of Rajasthan are a social
taboo and Kriti faces a lot of discrimination in her tailoring/sewing job in
terms of low pay and unruly comments from anyone who sees her working.
She also faces a lot of pressure from her in-laws to get pregnant and give
them a grandson, but she is unsure if she is ready for motherhood.\\
Daily routine:\\
– 6:00 - wakes up, finishes morning chores by 6:30\\
– 6:30 - prepares breakfast for the family\\
– 7:00 - 9:00 - engaged in house chores like laundry and cleaning\\
– 9:00 - 12:00 - Sewing business\\
– 12:00 - 1:00 - Serves lunch\\
– 1:00 - 1:30 - Afternoon nap\\
– 1:30 - 3:30 - Helps husband in family farm, grows vegetables and
flowers in a small portion of the farm\\
– 3:30 - 5:00 - Sells vegetables and flowers on a thela\\
– 5:00 - 8:00 - Sewing business\\
– 8:00 - 10:00 - Serves dinner, tends to in-laws and washes dishes and
other chores\\
– 10:00 - Goes to bed\\
Spending:\\
– In-laws’ medication\\
– Sends some money to parents from as they are too old to work and
earn\\
– Vegetables, rice, maize, barley\\
– Seeds, fertilizer for her horticulture business\\
Earning:\\
– Below minimum wage from sewing job, Rs. 5000 per month
3\\
– Vending vegetables and flowers, Rs. 8000 - 10000 per month\\
Aspirations:\\
– Given her bright intellect, she aspires to complete her education in a
reputed institute\\
– She has developed an entrepreneurial mindset and wishes to take her
tailoring business to the nest level by owning her own shop. Social
taboos, pressure from in-laws and financial constraints have kept her
from doing it.\\
What does she lack?\\
– Opportunities: In a rural area, there are not many earning/learning
opportunities especially for a woman.\\
– Money: Studying or Starting a business takes money, and Kriti is
trying her best to save up, but always loses most of it to her husband.\\\\
\textbf{Conclusion:} The major problem with the young adults of the region is the lack of opportunities. The literacy rate in villages like Sikar have traditionally been high but the state has gradually withdrawn from providing jobs in general and in the education, health and engineering sector in particular.\\
Districts like Sri Ganganagar and Hanumangarh benefited from the green revolution a lot which led to the emergence of a wealthy agrarian class which urbanised its lifestyle but there was no change in the social and cultural values. This can be seen by the poor sex ratio of these districts. The people in these areas also spend a lot of moneys on rituals and paying dowries \\
The areas are also victims of a health crisis created by the overuse of chemicals in the production of food items, including vegetables and fruits.\\
The emergence of an expensive yet non-productive lifestyle on the one hand, and the state’s withdrawal from the job market on the other, has created a severe socio-economic crisis. There are a large number of educated yet unemployed youth in all these districts.
\end{document}
